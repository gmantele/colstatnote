\documentclass[11pt,a4paper]{ivoa}
\input tthdefs

\lstloadlanguages{SQL,XML}
\lstset{flexiblecolumns=true,tagstyle=\ttfamily,
  showstringspaces=False}

\definecolor{rtcolor}{rgb}{0.15,0.4,0.3}
\newcommand{\rtent}[1]{\texttt{\color{rtcolor} #1}}

\title{Towards Blind Discovery 2: Advanced Column Statistics}

\ivoagroup{Registry}

\author[https://wiki.ivoa.net/twiki/bin/view/IVOA/MarkusDemleitner]{Demleitner, M.}

\editor{Demleitner, M.}

% \previousversion[????URL????]{????Concise Document Label????}
\previousversion{This is the first public release}
       

\begin{document}
\begin{abstract}
This note proposes extensions to the Virtual Observatory's registry
ecosystem to make advanced column statistics available for data
discovery.  The guiding use case here is to enable ``blind'' discovery,
i.e., search by physical properties of the data rather
than its textual description. For instance, people would constrain their
discovery query by ``more or less complete to $20^m$'' rather than
trying a full text search for ``deep survey''.
\end{abstract}


\section*{Conformance-related definitions}

The words ``MUST'', ``SHALL'', ``SHOULD'', ``MAY'', ``RECOMMENDED'', and
``OPTIONAL'' (in upper or lower case) used in this document are to be
interpreted as described in IETF standard RFC2119 \citep{std:RFC2119}.

The \emph{Virtual Observatory (VO)} is a
general term for a collection of federated resources that can be used
to conduct astronomical research, education, and outreach.
The \href{http://www.ivoa.net}{International
Virtual Observatory Alliance (IVOA)} is a global
collaboration of separately funded projects to develop standards and
infrastructure that enable VO applications.


\section{Introduction}


Finding data relevant to some piece of research, not by prior knowledge of
resources or by a full text search, but instead using constraints derived from
the data itself (“physical properties”) is called blind discovery. In some way,
it is the holy grail of data discovery. It particularly facilitates the re-use
of specialised observations that may be little known outside of a specific
community but may, for instance, go much deeper than the well-known global
surveys.

In Astronomy, the most important physical constraint for blind discovery
arguably is the the location in space, time, and spectrum.  Making these
properties usable for searches in the VO Registry was the focus of the
Roadmap for Space-Time Discovery in the VO Registry \citep{std:stcReg}.

With a proposed recommendation of VODataService 1.2 \citep{pr:VODS12}
under community review while this note is written, a major part of the
road towards STC-enabled discovery has been taken, and thus it is time
to tackle the next class of constraints useful in discovery: statistical
properties of tabular data.

Confining oneself to tabular (or, really, relational) data keeps the
problem tractable.
It also does not preclude blind discovery on
collections of array-like data (in particular observations like images,
spectra, time series).  Such collections are typically published to the
VO through tabular metadata collections, governed by standards like
ObsCore \citep{2017ivoa.spec.0509L} or EPN-TAP (still in WD at the time
of writing).  Statistics on the columns for such
relational representations can reasonably be expected to be useful for
many types of discovery of the data collections.  Hence,
this Note only deals with column statistics and leaves the mapping of
array collections to relational representations to other documents.

\subsection{Use Cases}
\label{sect:uc}

Use cases guiding the design of the present proposals include:

\begin{description}
\item[Deep Survey] Give me data for the M32 reaching 25 mag in the
infrared K band.

\item[High Redshift] I am looking for Galaxies with redshifts above 1.

\item[High Precision] I need a catalogue of proper motions with errors
below $0.2\,\rm mas/yr$.

\item[High Precision advanced] I need a catalogue of proper motions with
errors below $0.2\,\rm mas/yr$ at $15^{\rm m}$ in V.

\item[Calibrated] Where are flux-calibrated spectra for stars in
globular clusters?

\item[Planning] A service-spanning query engine wants to estimate how
many columns a query constrained with \verb|WHERE col<30| might yield.
\end{description}


\section{Prior Art}

In VO standards, column statistics have already been part of
VOTable version 1.1 \citep{2004ivoa.spec.0811O}, which defined the
\xmlel{MIN} and \xmlel{MAX} children of \xmlel{VALUES}, as well as
\xmlel{OPTION} elements to declare the values enumerated columns take.
VOTable allows qualifying \xmlel{VALUES} with a \xmlel{type} attribute;
for data discovery, only the case of \verb|actual| is interesting.

When the content model of \xmlel{vs:BaseParam} and its derivatives in
VODataService 1.1 \citep{2010ivoa.spec.1202P} was designed, VOTable's
\xmlel{FIELD} was obviously taken as a model, but the column statistics
was not taken over.

In consequence, the relational metadata schema for the
\verb|columns| table in TAP \citep{2019ivoa.spec.0927D} even in
version 1.1 does not include any column statistics.

Meanwhile, with the release of Gaia DR1 in 2016, Gr\'egory Mantelet
experimented with advanced column statistics on his Gaia mirror
\citep{data:arigaia}, exposing them both through the service's TAP
schema -- where extension columns are explicitly allowed and even shown
by clients like TOPCAT \citep{2005ASPC..347...29T} -- and the VOSI tables
endpoint, the latter using VODataService's extension hook of allowing
arbitrary non-namespace attributes on \xmlel{column}.

Mantelet's column statistics includes:

\begin{description}
\item[min\_value, max\_value] As in VOTable with \xmlel{type} \verb|actual|
\item[q1, median, q3] The three quartiles of the distribution of the
values in the column
\item[mean] The column values' mean value
\item[filling] The number of non-NULL values in the column
\end{description}


\section{Proposed Metadata}
\label{sect:metadata}

Clearly, the nature and amount of metadata to be processed strongly
depends on the use cases envisioned.  For instance, the use case
\emph{Planning} in sect.~\ref{sect:uc} will clearly profit from having
detailed histograms, in particular when one has to deal with multimodal
distributions.  The \emph{High precision advanced} use case would even
need the joint distribution of, in this case, proper motion errors and
magnitudes.

On the other hand, complex metadata statistics are hard to generate,
distribute and use.  It is therefore desirable to identify the
minimal set of metadata that already enables the most important use
cases.

An important distinction here is between continuous and discrete
variables, where the limit between the two is not a principled one in
digital computers.  For the present purposes, we would propose as a
practical definition: a column is discrete if an extra value
needs some explanation (``new object observed'', ``new pipeline now does
flux calibration'').


\subsection{Metadata for Continuous Variables}

Varying Mantelet's metadata set, we propose the following items for
non-enumerated columns:

\begin{description}
\item[min\_value, max\_value] As in Mantelet.  While the items as such are
typically rather unreliable, as for most astrophysically relevant
distributions the tails are sparse, they have obvious interpretations,
allow an estimate of the tailedness of the distribution, and they are
part of the VOTable model.  Hence, it seems prudent to retain them.
\item[percentile03, median, percentile97] the 3rd, 50th, and 97th
percentiles of the distribution of the values in the column.
\item[fill\_factor] The ratio of non-NULL values to the number of rows
in the table.
\end{description}

Changes versus the Mantelet model include:

\begin{itemize}
\item We have dropped the mean, as we believe the median is a robust
alternative that has the additional advantage that it is stable against
common arithmetic operations (e.g., going from parallax to distance).
\item We have replaced the quartiles with the 3rd and 97th percentiles.
This is because we believe that in general having robust estimates for
the extremes of a distribution is more important of data discovery than
understanding its behaviour around its modus.  3 and 97 were used
because for a Gauss distributed random variable, this corresponds to
roughly $2\sigma$ (or 95\% of the values), which seems a good and
intuitive handle for ``characteristic data''.
\item We are dividing Mantelet's filling by the number of rows in the
table in question.  The rationale here is that this number is more
likely to be meaningful in discovery than the raw number of non-NULL
values in a scenario like, say, ``filter out tables that only rarely
have redshifts''.
\end{itemize}

\subsection{Metadata for Discrete Variables}

The prior art for describing discrete variables in the VO is VOTable,
where \xmlel{VALUES} can have \xmlel{OPTION} children, which in turn
have \xmlel{name} and \xmlel{value} attributes.  Such option elements
can be used recursively, so that, in the word of the specification, 
``a hierarchy of keyword-values pairs can be associated with each
field''.

Column statistics on discrete variables are necessary in our use cases
\emph{Calibrated} -- for instance, on columns containing the SSA
calibration fields or the ObsCore \verb|o_calib_status| -- and
\emph{Planning}.  In both, no need for ``hierarchies of keyword-value
pairs'' is discernable.

An equivalent of \xmlel{name}, on the other hand, might have a use in
data discovery, as it could be a structured indication to humans what a
given value might mean.  At this point, however, it does not seem to
offer a substantial benefit over indicating the meaning of a parameter
and its special values in the human-readable description for a column.

Missing in VOTable while useful at least for \emph{Planning}, possibly
even for \emph{Calibrated} (say, to filter out data collections where
only very few datasets have been flux-calibrated), is the frequency of a
given discrete value.  As for the fill factor in the continuous case,
giving relative rather than absolute frequencies saves divisions that may
be difficult in discovery queries.

Hence, the model for discrete variables would be a sequence of (value,
relative-frequency) pairs.

While metadata for continuous variables as proposed here has
constant length, metadata for discrete variables modelled in this way
can arbitrarily grow in
size.  For instance, by our definitions, source identifiers make up the
domain of a discrete variable; flooding the registry with the source
identifiers in a large catalogue almost certainly is not desirable.

It is therefore likely that advice of the kind ``avoid giving
distributions over more than a few dozen elements'' will be necessary in
an actual specification.


\section{Required Standards Work}

\subsection{VODataService and TAP}

To communicate the extra metadata to the Registry, VODataService
tablesets would need to be extended.  A relatively un-intrusive change
to capture continuous-variable statistics would be to add a type
\xmlel{vs:Stats} to VODataService that has the proposed metadata from
sect.~\ref{sect:metadata} as attributes and would allow arbitrary
non-namespace elements as children to facilitate prototyping further
extensions. An element of this type would then be added to
\xmlel{vs:BaseParam} as a \xmlel{stats} child with occurrence 0 or 1.

Typing is a problem in this definition; while ``continuous'' variables
more or less by definition are numeric (where integers, for instance in
counts, are explicitly included), we believe other sortable types should
not be excluded needlessly.  Hence, we would propose to have all
attributes from the domain of the variable\footnote{With the current
definitions, that is all of them except \emph{fill\_factor}.} of type
\xmlel{xs:token}  and define that their values are to be parsed by
VOTable rules using the type information of the associated BaseParam
(where we might need to comment how this rule should be applied when
non-VOTable type systems are in use -- or limit the element to
VOTableType-d Params).

For discrete-variable statistics, this \xmlel{vs:Stats} child would
probably need to grow children somewhat like \xmlel{OPTION} in VOTable.
Because the content model is rather different, it probably should not be
called ``option'', though, but perhaps rather \xmlel{value}, with an
optional attribute \xmlel{freq}.  As in VOTable, simultaneous use of
discrete and continuous statistics should probably be discouraged at
first, as it may be that this mechanism could actually be re-used to
communicate actual histograms.

The types of discrete variables are essentially unconstrained, and hence
the value attribute would again be of type \xmlel{xs:token} with VOTable
syntax.

As the extra statistics enter VODataService, for symmetry they should
probably also be added in the TAP schema.  Statistics for discrete
variables would require another TAP schema table or the use of two
array-valued columns.

In the database table, where on-the-fly conversion by column type is at
least much harder, the typing question becomes a lot more pressing, in
particular because clients will want to use the statistics columns in
comparisons.  We
suggest that for purposes of the experiment, the statistics columns
would all be floating point-typed, and annotations for non-numeric
columns for now are not represented in the interoperable TAP schema.

With similar considerations, we would defer any decision in the question
of representing statistics of discrete-valued columns to when use cases
for them are better understood.

\subsection{RegTAP}

To exploit the new metadata in discovery, they need to be included in
RegTAP's table schema.  For the simple numeric statistics, this could be
done with a plain extension of \rtent{rr.table\_column}.  As in the
considerations of the TAP schema, this would leave out statistics on
non-numeric and in particular discrete values.

An attractive alternative, in particular as long as only a small
fraction of columns will come with statistics, might be separate tables,
perhaps separated into statistics for continuous numeric columns
(\rtent{rr.stat\_num}), for continuous other columns (where their string
representation would be used for values; \rtent{rr.stat\_token},
perhaps), and for discrete columns (where again string representations
would be held in RegTAP; \rtent{rr.stat\_discrete}).

Note that while in particular metadata on discrete variables might be
interesting for use in declaring input parameters (in
\xmlel{vs:paramHTTP}) in VOSI capability responses, at this point we do
not see a sufficient scenario for offering such information in RegTAP's
discovery-oriented schema.  Hence, the statistics columns would hold
foreign keys into \rtent{rr.table\_column} and could only be used for
them (and not for \rtent{rr.intf\_param}).


\section{Implementation Status}

In order to assess the feasibility and usability of the proposal, the
generation and communication of statistics for the continuous variables
have been implemented in DaCHS using a provisional scheme compatible
with current VODataService, again inspired by Mantelet's schema; the
data is not available in the TAP schema so far.

We have written a VOResource extension schema for the namespace
\nolinkurl{http://dc.g-vo.org/ColStats-1}, canonical prefix
\verb|g-colstat|, which defines namespaced attributes for use
with \xmlel{vs:BaseParam} (which allows arbitrary attributes from
outside the namespace already).  This is not intended as a permanent
extension but as a transitional device until the \xmlel{stats} element
proposed above (or something comparable) enters VODataService.

Since version 2.3, the DaCHS server package contains support for
generating such information. In its scheme, operators run 
\verb|dachs limits| 
on tables to obtain statistics, which is then automatically
inserted into the resource records produced.  Most of the records on
the OAI-PMH endpoint of GAVO's Heidelberg data
centre\footnote{\url{http://dc.g-vo.org/oai.xml}} feature these new
statistics.

On the network of RegTAP services at \url{http://reg.g-vo.org/tap},
these
statistics are re-published in a table \rtent{rr.g\_num\_stat},
containing, at the time of writing, about 1000 records; the prefix
\verb|g_| should again indicate that this is a provisional extension.

No implementation has been done for statistics of discrete
columns yet.

Other data center operators are cordially invited to try the proposed
mechanisms; this will certainly be helpful for evaluating the proposals.


\section{Sample Queries}

The most obvious use case, ``give me resources with data reaching a
certain depth'', could be satisfied with a query like

\begin{lstlisting}[language=SQL]
SELECT ivoid, table_name, percentile97
FROM rr.table_column
NATURAL JOIN rr.res_table
NATURAL JOIN rr.g_num_stat
WHERE ucd='phot.mag;em.ir.k'
  AND percentile97>15
\end{lstlisting}

where obviously further constraints could be added.  Our \emph{Deep
Survey} use case, for instance, would also constrain the position:

\begin{lstlisting}[language=SQL]
SELECT ivoid, table_name, percentile97
FROM rr.table_column
NATURAL JOIN rr.res_table
NATURAL JOIN rr.g_num_stat
NATURAL JOIN rr.stc_spatial
WHERE ucd='phot.mag;em.ir.k'
  AND percentile97>15
  AND 1=CONTAINS(POINT(10.6742,40.8652), coverage)
\end{lstlisting}

\section{Implementation Notes}

Most database systems have built-in mechanisms for obtaining
quantiles.  In Postgres, for instance, quantiles can be computed as in

\begin{lstlisting}[language=SQL]
SELECT 
  percentile_cont(ARRAY[0.03, 0.5, 0.97]) 
    WITHIN GROUP (ORDER BY kmag) AS k_stats,
  percentile_cont(ARRAY[0.03, 0.5, 0.97]) 
    WITHIN GROUP (ORDER BY jmag) AS j_stats,
  percentile_cont(ARRAY[0.03, 0.5, 0.97]) 
    WITHIN GROUP (ORDER BY hmag) AS h_stats
FROM twomass.data
\end{lstlisting}

where the percentiles recommended here are then returned as arrays.

Starting with version 3.28, SQLite has a window function NTILE that could be
used like this (for statistics of a column \verb|v| in a table
\verb|vals|):

\begin{lstlisting}[language=SQL]
SELECT 
  AVG(v), bucket 
FROM (
  SELECT v, NTILE(100) OVER (ORDER BY v) AS bucket 
  FROM vals) 
GROUP BY bucket 
HAVING bucket in (3, 50, 97)
\end{lstlisting}

where, compared to Postgres' facilities it is a bit harder to obtain
statistics for multiple columns in one query with NTILE.

For large tables, statistics gathering becomes expensive.  DaCHS hence
samples progressively smaller parts of tables, starting at 50\% at 10
Megarows and ending up at 5\% at 10 Gigarows.  This is ad-hoc at the
moment and, of course, depends on a reasonably non-biased sampling.

\bibliography{ivoatex/ivoabib,ivoatex/docrepo,local.bib}

\end{document}
